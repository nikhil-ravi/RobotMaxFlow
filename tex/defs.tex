\bibliographystyle{plainnat}

%\usepackage{xspace}
%\usepackage{beamerthemesplit} %Activate for custom appearance

\usepackage{xcolor} %for color
\usepackage{xmpmulti}
%\usetheme{Air}
\usefonttheme{professionalfonts}
\usepackage{thumbpdf}
\usepackage{wasysym}
\setbeamersize{text margin left=6mm}
\setbeamersize{text margin right=6mm}
\usepackage{upgreek}
\usepackage{ucs}
%\usepackage[utf8]{inputenc}
\usepackage{pgf,pgfarrows,pgfnodes,pgfautomata,pgfheaps,pgfshade}
\usepackage{verbatim}
\usepackage{empheq}
\newcommand*\widefbox[1]{\fbox{\hspace{2em}#1\hspace{2em}}}

\definecolor{darkblue}{rgb}{0,0,0.7}
\definecolor{dred}{rgb}{0.8,0.25,0.2}

\def\blue{\color{blue}}
\def\red{\color{dred}}
\def\darkblue{\color{darkblue}}


\setbeamertemplate{blocks}[rounded][shadow=true]
\setbeamercolor{block title}{fg=white,bg=dred}
\setbeamercolor{block body}{fg=black,bg=gray!5}


%Sergio's %%%%%%%%%%

%\usetheme{Boadilla}
%\usecolortheme{seahorse}
%\usefonttheme{professionalfonts}
%\usepackage{flashmovie}

%\usepackage[english]{babel}
%\usepackage[utf8x]{inputenc}

%\usepackage[latin1]{inputenc}

%\usepackage{times}

%\usepackage[T1]{fontenc}
\usepackage{graphicx}
\usepackage{epsfig,color,cases,url,soul}
%%%%%%%%%%%%%%%%%%%%%%%%%%%%%%%%%%%%%%%%%%%%%%%
\usepackage{amsmath,amssymb,bm,mathtools}
%\usepackage{algorithmicx,algpseudocode,algpascal}
\usepackage{xmpmulti,url,tikz}
%\usepackage{multimedia}
%\usepackage{fancyhdr}
%\usepackage{beamerouterthemesmoothbars}
% outer themes include default, infolines, miniframes, shadow, sidebar, smoothbars, smoothtree, split, tree
%\useoutertheme[subsection=false]{smoothbars}
%\useoutertheme[subsection=false,footline=authorinstitute]{smoothbars}
%\useoutertheme{smoothbars}

\setbeamercovered{transparent}

\definecolor{hughesblue}{rgb}{.9,.9,1} %A blue I like to use for highlighting, matches Hughes Hallet's book

\definecolor{orange}{rgb}{.953 ,.502 ,.04}
\setul{0.4ex}{0.2ex}
%\setulcolor{orange}
\setulcolor{pacificblue}

\usepackage{blindtext}
\usepackage{hyperref}
\hypersetup{
    colorlinks=true,
    linkcolor=blue,
    filecolor=magenta,
    urlcolor=cyan,
    pdftitle={Overleaf Example},
    pdfpagemode=FullScreen,
}

\makeatletter
\newcommand\SoulColor{%
    \let\set@color\beamerorig@set@color
    \let\reset@color\beamerorig@reset@color}
\makeatother


\newcommand{\Integer}{\mathbb{Z}}
\newcommand{\Natural}{\mathbb{Z}_{\geq 0}}
\newcommand{\Naturalstar}{\mathbb{Z}_{> 0}}
\newcommand{\Real}{\mathbb{R}}
\newcommand{\Complex}{\mathbb{C}}
\newcommand{\hilbert}{\mathcal{H}}
\newcommand{\BigHilbert}{\bm{\mathcal{H}}}
\newcommand{\innprod}[2]{\langle{#1},{#2}\rangle}
\newcommand{\ginnprod}[2]{\langle\!\langle{#1},{#2}\rangle\!\rangle}
\newcommand{\norm}[1]{\|{#1}\|}
\newcommand{\gnorm}[1]{|\!|\!|{#1}|\!|\!|}
\newcommand{\expect}{\mathbb{E}}

\newcommand{\gr}{\selectlanguage{greek}}

%\newcommand{\red}{\color{myred}}
%\newcommand{\blue}{\color{myblue}}
%\newcommand{\black}{\color{myblack}}

\DeclareMathOperator{\argmin}{arg\,min}
\DeclareMathOperator{\card}{card}
\DeclareMathOperator{\relinterior}{ri}
\DeclareMathOperator{\interior}{int}
\DeclareMathOperator{\boundary}{bdry}
\DeclareMathOperator{\linspan}{span}
\DeclareMathOperator{\trace}{trace}
\DeclareMathOperator{\vect}{vect}
\DeclareMathOperator{\modulo}{mod}
\DeclareMathOperator{\conv}{conv}
\DeclareMathOperator{\sign}{sgn}
\DeclareMathOperator{\diag}{diag}
\DeclareMathOperator{\lev}{lev_{\leq 0}}
\DeclareMathOperator{\leva}{lev_{\leq\alpha}}
\DeclareMathOperator{\Fix}{Fix}
\DeclareMathOperator{\sinc}{sinc}
\DeclareMathOperator{\BP}{BP}
\DeclareMathOperator{\abs}{abs}
\DeclareMathOperator{\supp}{supp}
\DeclareMathOperator{\graph}{gph}